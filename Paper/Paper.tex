\documentclass[acmsmall]{acmart}

\AtBeginDocument{%
  \providecommand\BibTeX{{%
    Bib\TeX}}}
\setcopyright{acmlicensed}
\copyrightyear{2024}

\acmYear{2024}
\acmDOI{cybersecurity.tomcruise}
\acmJournal{TISSEC}
\acmVolume{420}
\acmNumber{69}
\acmArticle{3.14}
\acmMonth{1}

\setcopyright{acmlicensed}
\copyrightyear{2024}
\begin{document}

\title{Historic Cybersecurity Breaches Caused by a Lack of Physical Security}

\author{Isaac Boaz}
\email{boazi@wwu.edu}
\email{isaac.k.boaz@gmail.com}
\affiliation{%
  \institution{Western Washington University}
  \city{Seattle}
  \state{Washington}
  \country{USA}
}

\renewcommand{\shortauthors}{Boaz}

\begin{abstract}
  Cybersecurity requires many layers and aspects of security to achieved.
  This article analyzes and reviews historical occurences where a lack of
  physical security lead to a lack of cybersecurity. In addition to introduction,
  an analysis and review of such occurences will be presented.
\end{abstract}

\keywords{Cybersecurity, Security, Software, Physical, Breaches}

\begin{CCSXML}
  <ccs2012>
  <concept>
  <concept_id>10003456.10003462.10003561.10003563</concept_id>
  <concept_desc>Social and professional topics~Network access restrictions</concept_desc>
  <concept_significance>300</concept_significance>
  </concept>
  <concept>
  <concept_id>10003456.10003462.10003561.10003566</concept_id>
  <concept_desc>Social and professional topics~Universal access</concept_desc>
  <concept_significance>100</concept_significance>
  </concept>
  <concept>
  <concept_id>10003456.10003462.10003574.10003000</concept_id>
  <concept_desc>Social and professional topics~Social engineering attacks</concept_desc>
  <concept_significance>500</concept_significance>
  </concept>
  <concept>
  <concept_id>10003456.10003462.10003574.10003578</concept_id>
  <concept_desc>Social and professional topics~Malware / spyware crime</concept_desc>
  <concept_significance>100</concept_significance>
  </concept>
  <concept>
  <concept_id>10002978.10002991</concept_id>
  <concept_desc>Security and privacy~Security services</concept_desc>
  <concept_significance>500</concept_significance>
  </concept>
  <concept>
  <concept_id>10002978.10003006</concept_id>
  <concept_desc>Security and privacy~Systems security</concept_desc>
  <concept_significance>500</concept_significance>
  </concept>
  </ccs2012>
\end{CCSXML}

\ccsdesc[500]{Security and privacy~Security services}
\ccsdesc[500]{Security and privacy~Systems security}
\ccsdesc[300]{Social and professional topics~Network access restrictions}
\ccsdesc[500]{Social and professional topics~Social engineering attacks}
\ccsdesc[100]{Social and professional topics~Universal access}
\ccsdesc[100]{Social and professional topics~Malware / spyware crime}

\maketitle

\section{Introduction}
In the world of software and programs, many people in the cybersecurity
field focus on the security and practices used in their software and programs.
However, one of the most well-known ideas for security is the idea that
your security is only as strong as your weakest link. Just as electricity
follows the path of least resistance, hackers and cybercriminals need only
find a single weak point in your security to gain access to the rest of
your system.

Physical security is a part of cybersecurity. The act of protecting and
securing IT assets, data, and resources from physical attacks is a part
of cybersecurity. This article will review and analyze both fictional and
non-fictional occurences where a lack of physical security lead to a lack
or breach of cybersecurity.

\section{Basic Physical Security}
Regardless of what a device is doing, it is always important to understand
the physical environment and security that the device is in. Even the
best anti-virus software cannot be protected from a malicious actor
uninstalling it. Physical access could mean a man in the middle attack,
power outages, or even a keylogger being installed.

Some basic physical security practices include:
\begin{itemize}
  \item Locking a device when not in use
  \item Restricting physical access to those that need it
  \item Locking the doors and windows of a building
  \item Scanning and limiting who can enter a building
  \item Ensuring that all devices are up to date
\end{itemize}

Ultimately, physical security can be just as important and just as in-depth
as cybersecurity.

\section{Google's Plasma Globe \texorpdfstring{\cite{Google01}}{}}
One of the best ways to test your security is by doing it yourself. Google
went about this by creating a "Red Team", a team of white-hat hackers that
were tasked with trying to hack Google. Though this task certainly includes
doing software-based attacks, one of the successful attacks was actually
a physical attack.

\subsection{Attack}
The Red Team designed a Plasma Globe that had a USB cable attached to it for power.
Only five plasma globes were created with a malicious modification to them.
The Red Team added an internal chip that would emulate a USB keyboard when plugged
in to a computer. After approximately 5-10 minutes \cite{lcamtuf}, the device
would quickly type out a command to download a larger payload from the internet.
In order to give the device to Google's employees, the Red Team disguised the
device as an anniversary gift from Google (based off of LinkedIn).

While not everyone that received the device plugged it in, it only took one of
Google's employees to plug it in for the attack be successful. Upon successful
execution of the script, the Red Team had access to the employee's credentials.
Through this, they were able to send emails to other employees which had higher
privileges. Through these emails with attachments, the Red Team was able to
gain access to Google's Glasses internal documentation.

Finally, the Red Team attempted to obtain a physical build of the Google Glasses.
Unfortunately, due to a few typos, the Red Team caused a few red flags to be raised,
and instead of going to pick up the build, they were met with Google's Chief Security
Officer.

\subsection{Analysis}
What ultimately allowed this attack to be successful was improper training, understanding,
and awareness of physical security. Google's employees were not trained to be aware
of the potential dangers of plugging in a USB device was given to them.

This attack also highlights the importance of physical security. If Google employees
were not allowed to bring in unauthorized external devices, this attack would not
have been successful.

\subsection{Mitigation}
Google learned from this breach and created a software that would listen to extremely
fast keystrokes from USB devices and block them. This software was then installed
on all of Google's computers, and publicly released \cite{Google02}.

While a software solution is a good start, it is not a complete solution. A theoretical
slow USB device could still be used to bypass this software. A better solution would
be to train employees to be aware of the dangers of plugging in unauthorized devices.

\section{Now You See Me 2}
While not a real-life example, the movie Now You See Me 2 \cite{NowYouSeeMe2} presents
a potential example of a physical security breach. In the movie, the main characters
are able to breach a company's phone presentation by disguising as server maintenance,
quickly convincing security that the original server maintainer was crazy, and
installing a device that would allow them to control the presentation.

\subsection{Attack}
While very entertaining, the attack in the movie is not entirely realistic. Quick uniform
changes and pretending that you belong was presented to seem entirely plausible. Regardless,
the important parts of the attack were that the server room was easily accessible by all staff
without only one layer of security. The single layer was a wired door that was left open
by the original server maintainer.

Once the main character was in the server room disguised, he took a portable camera that could
immediately print out a picture and pretended that it was "picture day" for the maintainer.
The maintainer was not fooled by the lies, however, he was too confused and insisted that the
character should not be there. Regardless, The main character printed out the picture and
placed it on a badge for a mental facility.

Upon calling for security, the main character was able to convince the security that the
maintainer was crazy and that he was the real maintainer. The main character was then able
to install a device that would allow him to control the presentation.

\subsection{Analysis}
This attack was successful because of a lack of physical security. The server room was
easily accessible by all staff and only had a single layer of security. The maintainer
also left the door open, allowing the main character to easily enter the room.
With no physical keycard, identification, or other security, the main character was
able to easily enter the room.

\subsection{Mitigation}
Ideally, the server room would have had a physical keycard, identification, or other
security that would have prevented the main character from entering the room. The
maintainer should have also locked the door behind him, preventing the main character
from entering the room. Lastly, ideally the security guards would have been trained
to be aware of social engineering and to not be easily convinced by lies.

\section{Mission Impossible \texorpdfstring{\cite{missionimpossible}}{}}
A more realistic example of a physical security breach is in the movie
Mission Impossible. In the movie, Ethan Hunt (played by Tom Cruise) is tasked
with stealing the NOC list from a CIA building. The NOC list is a list of all
CIA agents that are currently undercover. The CIA building is heavily guarded
and has many layers of security.

\subsection{Attack}
In the movie, we are shown the many physical layers of security that the CIA
building has. The building has a security guard, a security camera, a keycard
reader, and a physical key. Access to the NOC list is also restricted to only
the physical computer that the NOC list is on. The server room employs many
measures and monitors to prevent unauthorized access.

Specifically, the server room has pressure-sensitive floors, a temperature monitor,
sound sensor, and lasers above the ventilation system. All of these measures are disabled
while an authorized person is in the room, and re-enabled when they leave.

Ethan Hunt goes through the ventilation system, uses a mirror to reflect the
lasers, and is slowly lowered to the floor by a wire. He then hacks into the
computer and steals the NOC list.

\subsection{Analysis}
Here we can see that the CIA building has many layers of physical security.
Despite all of these layers, Ethan Hunt was still able to breach the security.
Unfortunately, the CIA building failed to account for plot armor, which is
a classic mistake in the field of cybersecurity.

\subsection{Mitigation}
Realistically, the CIA building should have had a more secure ventilation system;
we see that the team was able to get into the ventilation system (and the building itself)
by disguising as firefighers. The CIA did not have proper protocols or procedures
for fire alarms or drills, which allowed the team to easily enter the building.

Additionally, requiring an authorized person to be in the room to access the
computer would be a good requirement.

\section{Conclusion}
We have seen examples of physical security and how attackers may navigate around
it. Being aware of all potential entrypoints and the human element of security is important
to keep in mind when designing a secure system. Security exploits could stem from the
design of a building (such as a server room being too accessible) to the training of
its employees.

Ensuring that all employees are trained to be aware of physical security and the dangers
of social engineering is important. Additionally, ensuring that all employees are trained
to be aware of the dangers of plugging in unauthorized devices is important. Having properly
defined protocols and procedures for unlikely events is also important.
Companies should internally test their security- GitLab for example similarly has a
"Red Team" that is tasked with trying to hack GitLab \cite{GitLab}.

\begin{acks}
  I would like to thank Google for their Plasma Globe attack, and the creators of
  Mission Impossible for their interesting portrayal of physical security.
\end{acks}

\bibliographystyle{ACM-Reference-Format}
\bibliography{bib}
\end{document}
\endinput