%%
%% This is file `sample-acmlarge.tex',
%% generated with the docstrip utility.
%%
%% The original source files were:
%%
%% samples.dtx  (with options: `acmlarge')
%% 
%% IMPORTANT NOTICE:
%% 
%% For the copyright see the source file.
%% 
%% Any modified versions of this file must be renamed
%% with new filenames distinct from sample-acmlarge.tex.
%% 
%% For distribution of the original source see the terms
%% for copying and modification in the file samples.dtx.
%% 
%% This generated file may be distributed as long as the
%% original source files, as listed above, are part of the
%% same distribution. (The sources need not necessarily be
%% in the same archive or directory.)
%%
%%
%% Commands for TeXCount
%TC:macro \cite [option:text,text]
%TC:macro \citep [option:text,text]
%TC:macro \citet [option:text,text]
%TC:envir table 0 1
%TC:envir table* 0 1
%TC:envir tabular [ignore] word
%TC:envir displaymath 0 word
%TC:envir math 0 word
%TC:envir comment 0 0
%%
%%
%% The first command in your LaTeX source must be the \documentclass
%% command.
%%
%% For submission and review of your manuscript please change the
%% command to \documentclass[manuscript, screen, review]{acmart}.
%%
%% When submitting camera ready or to TAPS, please change the command
%% to \documentclass[sigconf]{acmart} or whichever template is required
%% for your publication.
%%
%%
\documentclass[acmlarge]{acmart}

%%
%% \BibTeX command to typeset BibTeX logo in the docs
\AtBeginDocument{%
  \providecommand\BibTeX{{%
    Bib\TeX}}}

%% Rights management information.  This information is sent to you
%% when you complete the rights form.  These commands have SAMPLE
%% values in them; it is your responsibility as an author to replace
%% the commands and values with those provided to you when you
%% complete the rights form.
\setcopyright{acmlicensed}
\copyrightyear{2024}
% \acmYear{2018}
% \acmDOI{XXXXXX.XXXXXXX}


%%
%% These commands are for a JOURNAL article.
% \acmJournal{POMACS}
% \acmVolume{37}
% \acmNumber{4}
% \acmArticle{111}
% \acmMonth{8}

%%
%% Submission ID.
%% Use this when submitting an article to a sponsored event. You'll
%% receive a unique submission ID from the organizers
%% of the event, and this ID should be used as the parameter to this command.
%%\acmSubmissionID{123-A56-BU3}

%%
%% For managing citations, it is recommended to use bibliography
%% files in BibTeX format.
%%
%% You can then either use BibTeX with the ACM-Reference-Format style,
%% or BibLaTeX with the acmnumeric or acmauthoryear sytles, that include
%% support for advanced citation of software artefact from the
%% biblatex-software package, also separately available on CTAN.
%%
%% Look at the sample-*-biblatex.tex files for templates showcasing
%% the biblatex styles.
%%

%%
%% The majority of ACM publications use numbered citations and
%% references.  The command \citestyle{authoryear} switches to the
%% "author year" style.
%%
%% If you are preparing content for an event
%% sponsored by ACM SIGGRAPH, you must use the "author year" style of
%% citations and references.
%% Uncommenting
%% the next command will enable that style.
%%\citestyle{acmauthoryear}


%%
%% end of the preamble, start of the body of the document source.
\begin{document}


%   	An investigation of times that a lack in good physical security practices led to a cyber security issue

%%
%% The "title" command has an optional parameter,
%% allowing the author to define a "short title" to be used in page headers.
\title{Historic Cybersecurity Breaches Caused by a Lack of Physical Security}

%%
%% The "author" command and its associated commands are used to define
%% the authors and their affiliations.
%% Of note is the shared affiliation of the first two authors, and the
%% "authornote" and "authornotemark" commands
%% used to denote shared contribution to the research.
\author{Isaac Boaz}
\email{isaac.k.boaz@gmail.com}

%%
%% By default, the full list of authors will be used in the page
%% headers. Often, this list is too long, and will overlap
%% other information printed in the page headers. This command allows
%% the author to define a more concise list
%% of authors' names for this purpose.
\renewcommand{\shortauthors}{Boaz}

%%
%% The abstract is a short summary of the work to be presented in the
%% article.
\begin{abstract}
  Cybersecurity requires many layers and aspects of security to achieved.
  This article analyzes and reviews historical occurences where a lack of
  physical security lead to a lack of cybersecurity. In addition to introduction,
  an analysis and review of such occurences will be presented.
\end{abstract}

%%
%% Keywords. The author(s) should pick words that accurately describe
%% the work being presented. Separate the keywords with commas.
\keywords{Cybersecurity, Security, Software, Physical, Breaches}

%%
%% This command processes the author and affiliation and title
%% information and builds the first part of the formatted document.
\maketitle

\section{Introduction}
In the world of software and programs, many people in the cybersecurity
field focus on the security and practices used in their software and programs.
However, one of the most well-known ideas for security is the idea that
your security is only as strong as your weakest link. Just as electricity
follows the path of least resistance, hackers and cybercriminals need only
find a single weak point in your security to gain access to the rest of
your system.

Physical security is a part of cybersecurity. The act of protecting and
securing IT assets, data, and resources from physical attacks is a part
of cybersecurity. This article will review and analyze historical
occurences where a lack of physical security lead to a lack or even breach
of cybersecurity.

\section{Basic Physical Security}
Regardless of what a device is doing, it is always important to understand
the physical environment and security that the device is in. Even the
best anti-virus software cannot be protected from a malicious actor
uninstalling it. Physical access can mean a man in the middle attack,
power outages, or even a keylogger being installed. 

Some basic physical security practices include:
\begin{itemize}
  \item Locking a device when not in use
  \item Restricting physical access to those that need it
  \item Locking the doors and windows of a building
  \item Scanning and limiting who can enter a building
\end{itemize}

\section{Google's Plasma Globe \texorpdfstring{\cite{Google01}}{}} 
One of the best ways to test your security is by doing it yourself. Google
went about this by creating a "Red Team", a team of white-hat hackers that
were tasked with trying to hack Google. Though this task certainly includes
doing software-based attacks, one of the successful attacks was actually
a physical attack.

\subsection{Attack}
The Red Team designed a Plasma Globe that had a USB cable attached to it for power.
Only five plasma globes were created with a malicious modification to them.
The Red Team added an internal chip that would emulate a USB keyboard when plugged
in to a computer. After approximately 5-10 minutes \cite{lcamtuf}, the device
would quickly type out a command to download a larger payload from the internet.
In order to give the device to Google's employees, the Red Team disguised the
device as an anniversary gift from Google (based off of LinkedIn).

While not everyone that received the device plugged it in, it only took one of
Google's employees to plug it in for the attack be successful. The Red Team
gained unauthorized access to the documentation on Google's Glass, which
at the time was heavily in development.

\subsection{Analysis}
What ultimately allowed this attack to be successful was improper training, understanding,
and awareness of physical security. Google's employees were not trained to be aware
of the potential dangers of plugging in a USB device was given to them.

This attack also highlights the importance of physical security. If Google employees
were not allowed to bring in unauthorized external devices, this attack would not
have been successful.

\subsection{Mitigation}
Google learned from this breach and created a software that would listen to extremely
fast keystrokes from USB devices and block them. This software was then installed
on all of Google's computers.

While a software solution is a good start, it is not a complete solution. A theoretical
slow USB device could still be used to bypass this software. A better solution would
be to train employees to be aware of the dangers of plugging in unauthorized devices.

\section{Now You See Me 2}
While not a real-life example, the movie Now You See Me 2 \cite{NowYouSeeMe2} presents
a potential example of a physical security breach. In the movie, the main characters

% \bibliographystyle{ACM-Reference-Format}
% \bibliographystyle{unsrt}
\bibliography{bib}
\end{document}
\endinput
%%
%% End of file `sample-acmlarge.tex'.
