\documentclass[acmsmall]{acmart}
\usepackage{fancyhdr} 
\usepackage{listings}

\settopmatter{printacmref=false} % Removes citation information below abstract
\renewcommand\footnotetextcopyrightpermission[1]{} % removes footnote with 
% conference information in first column

\AtBeginDocument{%
  \providecommand\BibTeX{{%
    Bib\TeX}}}

\begin{document}
\title{The design flaw of npm and how `everything' broke it}

\author{Isaac Boaz}
\email{boazi@wwu.edu}
\email{isaac.k.boaz@gmail.com}
\affiliation{%
  \institution{Western Washington University}
  \city{Seattle}
  \state{Washington}
  \country{USA}
}

\renewcommand{\shortauthors}{Boaz}

\begin{abstract}
  npm (Node Package Manager) is a popular package manager owned by GitHub that
  allows developers to share node packages and libraries \cite{npm}. This article will
  review and analyze the design flaw of npm and how it was broken by a single
  package, and will also review how GitHub responded to the incident and what
  actions they took in response.
\end{abstract}

\keywords{npm, github, design flaws, everything, left-pad}

\begin{CCSXML}
  <ccs2012> <concept> <concept_id>10011007.10011074.10011134</concept_id>
  <concept_desc>Software and its engineering~Collaboration in software
  development</concept_desc> <concept_significance>500</concept_significance>
  </concept> <concept> <concept_id>10003456.10003462.10003463</concept_id>
  <concept_desc>Social and professional topics~Intellectual
  property</concept_desc> <concept_significance>100</concept_significance>
  </concept> <concept>
  <concept_id>10003456.10003457.10003490.10003491.10003495</concept_id>
  <concept_desc>Social and professional topics~Systems analysis and
  design</concept_desc> <concept_significance>300</concept_significance>
  </concept> <concept>
  <concept_id>10003456.10003457.10003490.10003491.10003492</concept_id>
  <concept_desc>Social and professional topics~Project management
  techniques</concept_desc> <concept_significance>300</concept_significance>
  </concept> <concept>
  <concept_id>10003456.10003457.10003490.10003491.10003496</concept_id>
  <concept_desc>Social and professional topics~Systems
  development</concept_desc> <concept_significance>100</concept_significance>
  </concept> <concept>
  <concept_id>10003456.10003457.10003490.10003503</concept_id>
  <concept_desc>Social and professional topics~Software
  management</concept_desc> <concept_significance>300</concept_significance>
  </concept> <concept>
  <concept_id>10003456.10003457.10003490.10003507.10003508</concept_id>
  <concept_desc>Social and professional topics~Centralization /
  decentralization</concept_desc>
  <concept_significance>300</concept_significance> </concept> <concept>
  <concept_id>10003456.10003457.10003490.10003507.10003509</concept_id>
  <concept_desc>Social and professional topics~Technology audits</concept_desc>
  <concept_significance>500</concept_significance> </concept> </ccs2012>
\end{CCSXML}

\ccsdesc[500]{Software and its engineering~Collaboration in software development}
\ccsdesc[100]{Social and professional topics~Intellectual property}
\ccsdesc[300]{Social and professional topics~Systems analysis and design}
\ccsdesc[300]{Social and professional topics~Project management techniques}
\ccsdesc[100]{Social and professional topics~Systems development}
\ccsdesc[300]{Social and professional topics~Software management}
\ccsdesc[300]{Social and professional topics~Centralization / decentralization}
\ccsdesc[500]{Social and professional topics~Technology audits}

\maketitle

\fancyfoot{}
\thispagestyle{empty}

\section{Article Introduction}
Design flaws in software can sometimes lead to security breaches. More often
than not, it is these types of design flaws that are the most popularized.
However, sometimes a design flaw can lead to a catastrophic failure that was not
even intended (or initiated) by a bad actor. With over 1.3 million packages as
of April 14th, 2021 \cite{npmblog-stats}, npm is the world's largest software
registry \cite{aboutnpm,w3schools}. As a result, it has been heavily relied upon
by developers for years.

The npm registry is a critical piece of infrastructure for the entire JavaScript
community (and by extension, 99\% of websites \cite{w3techs}). Developers expect
that the npm registry will be reliable, secure, and complete. However, the
left-pad incident in 2016 \cite{npmblog-leftpad} and the everything package
incident in 2023 \cite{youtube-everything} have shown that the npm registry is
not as mature as developers had hoped.

\section{Historical Background}
\subsection{Introduction}
One of the most infamous events in the history of npm was the left-pad incident.
Azer Koçulu is a software developer who was working on a personal project called
kik \cite{qz}. Coincidentally, Kik was also the name of a company in Ontario,
Canada \cite{qz, crunchbase}. The company reached out to Koçulu requesting that
he change the name of his project. Koçulu refused, and the company proceeded to
file a complaint with npm \cite{qz}. npm (following its Dispute Resolutinon
Policy \cite{npm-dispute-policy}) then decided to transfer the name kik to the
company, which caused Koçulu to remove all of his packages from the npm
registry, stating

"I want all my modules to be deleted including my account, along with this
package. I don't wanna be a part of NPM anymore. If you don't do it, let me know
how do it quickly. I think I have the right of deleting all my stuff from NPM."
\cite{medium-mikeroberts}

\subsection{The left-pad package}
As a result of Koçulu's deciscion, all of his packages (including left-pad) were
removed from the npm registry. Despite being only 17 lines of code
\cite{github-leftpad-code}, millions of packages relied on left-pad
\cite{github-leftpad-dependency}. As a result, many packages that directly or
indirectly relied on left-pad broke, including React, Babel, and Atom
\cite{npmblog-leftpad}.

After an attempt to publish left-pad under a new version failed (due to some
major dependencies relying on the old version specificially, which was no longer
available), npm resulted to restoring the package from a backup
\cite{npmblog-leftpad}. The total downtime taken to restore the backup was 2.5
hours \cite{npmblog-leftpad}.

With the first ever unprecedented un-unpublishing of a package, npm was able to
restore the left-pad package and the packages that relied on it
\cite{npmblog-leftpad}.

\subsection{Mitigation and Prevention}
Once the left-pad incident was resolved, npm published a blog post detailing the
incident and the steps they planned to take to prevent it from happening again
\cite{npmblog-leftpad}.

`We will make it harder to un-publish a version of a package if doing so would
break other packages. We are still fleshing out the technical details of how
this will work. Like any registry change, we will of course take our time to
consider and implement it with care.' \cite{npmblog-leftpad}

npm ended up enacting two policies relating to unpublishing packages. The first
policy applied on packages that were less than 72 hours old, allowing
unpublishing as long as `no other packages in the npm Public Registry depend on
your package' \cite{npm-docs-unpublishing}.

The second policy applied to packages that were older than 72 hours, allowing
unpublishing as long as three conditions were met \cite{npm-docs-unpublishing}:
\begin{enumerate}
  \item no other packages in the npm Public Registry depend on it
  \item it had less than 300 downloads over the last week
  \item it has a single owner/maintainer
\end{enumerate}

With these new policies in place, npm hoped to prevent a similar incident from
happening again. However, these new policies lead towards a new unforseen design
flaw.

\section{The everything package}
\subsection{Everything's Creation}
In December 2023, a developer by the name PatrickJS decided to work on a package
called `everything'. He teamed up with Evan Boehs and a few other developers to
work on creating an npm package that would declare every package in the npm
registry as a dependency \cite{uncenter-blog-everything}. This invovled working
around obscure npm limitations, such as the maximum number of dependencies being
around 800 \cite{uncenter-blog-everything, youtube-everything}, and the maxiumum
package size being 10 MB \cite{uncenter-blog-everything, youtube-everything}.

In order to work around these limitations, PatrickJS and Evan Boehs decided to
chunk their dependencies into smaller packages
\lstinline|@everything-registry/chunk-0| through to \\
\lstinline|@everything-registry/chunk-4|. At the time there was approximately
2.5 million packages in the npm registry \cite{youtube-everything}, so
sub-chunking was also required \cite{youtube-everything}.

Unsurprisingly, npm also has a rate limit on how quickly packages can be
published \cite{npm-ratelimiting,github-npm-ratelimiting}. Thus, PatrickJS and
Evan Boehs used a GitHub Action workflow to publish the packages
\cite{youtube-everything,uncenter-blog-everything}.

\subsection{The Aftermath}
Upon publishing of the `everything' package in Dec 2023 \cite{npm-everything},
PatrickJS and Evan Boehs had (unintentionally) disabled unpublishing every
package ever on the npm registry \cite{youtube-everything}. This was an
unforseen consequence of the new unpublishing policies that npm had put in place
after the left-pad incident. Namely, \textit{every package was now a dependency
  of the `everything' package}, and thus, could not be unpublushed, regardless of
age. \lstinline|everything| itself could not be removed, as it had one
dependency: \lstinline|everything-else| \cite{npm-everything-else}.

This was also caused by the fact that the `everything' package depended on a `*'
version of all of its packages, which meant that it matched any version of each
of its packages. Even if developers published a new version of their packages,
the `everything' package would actually just depend on both the old and new
versions of the package \cite{uncenter-blog-everything}.

\subsection{Historic Related Incidents}
npm was no a stranger to packages and projects similar to `everything'.

\subsubsection{no-one-left-behind}
was an npm package that was created in 2018 that depended on 1,000 npm packages
\cite{npm-no-one-left-behind-dependencies}. npm managed to remove the package
before it caused significant damage \cite{npm-no-one-left-behind-security}. It
was then re-uploaded under a different name, this time with 33,000 dependencies
\cite{socket-everything}.

\subsubsection{hoarders}
was a package created by Josh Holbrook that was created near the conception of
npm (2012) that also depended on 11,000 packages at its inception
\cite{github-hoarders-dependencies}. Isaac Schluter (the founder of npm) talked
with the creator of the package and convinced him to unpublish it
\cite{github-hoarders}.  A few years later in Aug 19, 2021, Josh Holbrook found
a way to work around having a heavy load on npm.

`now, rather than installing the utilities as direct dependencies, hoarders
installs them lazily, on-the-fly. this change is practically seamless - simply
reach for the utility like you would normally and hoarders will do the rest! the
only requirements are npm and an internet connection.' \cite{github-hoarders}

\subsection{Developer's Response}
PatrickJS and Evan Boehs weren't aware of the consequences of their actions
until they realized that they had broken the npm registry
\cite{youtube-everything}. They reached out to npm and created an issue on the
GitHub repository documenting and explaining the issue
\cite{bleepingcomputer-everything}.

uncenter, a developer who assisted in the creation of the `everything' package,
wrote:

`We immediately reached out to GitHub; Patrick used his network and contacts to
speak to people at GitHub, and we sent multiple emails to the support and
security teams on NPM. Unfortunately, these events transpired over the holidays
and the NPM/GitHub teams were not responding (likely out of the office). We
continued to get harsh and rude comments from random people with a little too
much time on their hands... one person even wrote a 1400 word rant about the
unpublishing issue, despite us repeatedly telling them we could do nothing
further' \cite{uncenter-blog-everything}.

\section{GitHub's Response}
Due to this happening over the holidays, it was only until the end of day on Jan
2, 2024 that GitHub reached out to the developers and notified them that they
were aware of the problem \cite{uncenter-blog-everything}. The next day (Jan
3rd), GitHub flagged the repository and began removing the packages on npm.

\subsection{Review}
On Jan 4th, GitHub sent a follow-up email to the developers classifying the
project as against their terms of service, and that they would be removing the
repository \cite{uncenter-blog-everything}. GitHub removed the repository and
packages from npm.

\section{Design Flaw Analysis}
The `everything' incident exposed a fundamental flaw: npm's design allows
developers to create packages that depend on every other package in the
registry, or a large number of packages.

\subsection{Proposed Solutions}
There have been a few proposed solutions to the design flaw of npm. It appears
that GitHub does not have any immediate plans to change their policies judging
by their response to the `everything' package incident
\cite{uncenter-blog-everything}.

\subsubsection{uncenter} proposes `NPM should either a) prevent folks from
publishing packages with star versions in the package.json entirely, or b) don't
consider a dependent of a package if it uses a star version when tallying how
many packages depend on a package for unpublishing.'
\cite{uncenter-blog-everything}.

\subsubsection{PatrickJS} proposes `A) allow folks to unpublish when the
packages that depend on them use a "star" version, B) not permit star versions
in published packages going forward' \cite{bleepingcomputer-everything}

At its core, this is a fundamental design flaw of npm. The npm registry is
designed to be a public registry, and as such, it is difficult to prevent
developers from creating packages that depend on every other package in the
registry. This is a problem that is difficult to solve without fundamentally
changing the design of the npm registry.

\subsection{Other Potential Solutions}
\subsubsection{Rate Limiting}
One potential solution to this problem is to rate limit the number of packages
that a single package can depend on. This would prevent developers from creating
packages that depend on every other package in the registry.

Realistically, an initial package should not depend on more than a few hundred
other packages (especially not directly). This would be a good indicator that
the package is not well-designed, and would allow the npm team to review the
package before allowing it to be published.

The idea of rate-limiting the growth of a package's dependencies would require
careful consideration and implementation, as it could potentially break existing
packages that rely on a large number of dependencies.

\subsubsection{Dependency Analysis}
Another potential solution is to analyze the dependencies of a package before
allowing it to be published. If a package depends on every other package in the
registry (or a significant portion of it), it could be flagged for review by the
npm team.

Static code analysis could additionally be used to detect dependencies that are
not actually used by the package, and flag the package for review.

\subsubsection{Unpublishing Policies}
The unpublishing policies that npm put in place after the left-pad incident were
designed to prevent a similar incident from happening again. However, they ended
up creating a new design flaw that was exploited by the `everything' package.
npm could consider revising these policies to prevent a similar incident from
happening in the future.

While there may not be a perfect policy that works for every package, npm could
consider implementing a review process for new packages that are published to
the registry from new developer (accounts), or from developers that have not
published a package in a long time.

\subsubsection{Review Process}
npm could also consider implementing a review process for new packages that are
published to the registry and contain a large number of dependencies. This would
allow the npm team to catch packages like `everything' before they are published
to the registry.

Such a review process would require some form of automation and user reporting
system. Considering npm is free to use \cite{npm-tos}, it may not be feasible to
have sufficient staffing for this process.

\subsubsection{Accept the Flaw}
The route that GitHub seems to take (for now at least) is to accept the flaw and
mark packages like `everything' as against their terms of service
\cite{uncenter-blog-everything}. While this may have worked in the case of the
`everything' package, it is not a long-term solution to the problem. GitHub will
most likely have to automate some form of detection and removal of packages like
`everything' in the future.

\section{Conclusion}
The `everything' package incident was a catastrophic failure that was not
intended by the developers of the package. It was the result of a design flaw in
the npm registry. The incident highlights the need for better security and
reliability in the npm registry, and the need for better policies to prevent
similar incidents from happening in the future.

\section{Acknowledgements}
I would like to thank Theo Browne for introducing this recent incident to me by
the form of a YouTube video \cite{youtube-everything}. I would also like to
thank the developers of the `everything' package for sharing their story and
providing valuable insights into the incident.

\bibliographystyle{unsrtnat}
\bibliography{bib}

\appendix
\section{Appendix: Reading Resources}
\begin{itemize}
  \item Azer Koçulu himself wrote a blog post about the incident, which can be
        found at \cite{appendix-medium-azer}.
  \item The everything team created a website advertising the package, which can
        be found at \cite{appendix-everything-website}.
\end{itemize}

\end{document}
\endinput