\documentclass[acmsmall]{acmart}
\usepackage{fancyhdr} 
\usepackage{listings}

\settopmatter{printacmref=false} % Removes citation information below abstract
\renewcommand\footnotetextcopyrightpermission[1]{} % removes footnote with conference information in first column

\AtBeginDocument{%
  \providecommand\BibTeX{{%
    Bib\TeX}}}

\begin{document}
\title{The design flaw of npm and how `everything' broke it}

\author{Isaac Boaz}
\email{boazi@wwu.edu}
\email{isaac.k.boaz@gmail.com}
\affiliation{%
  \institution{Western Washington University}
  \city{Seattle}
  \state{Washington}
  \country{USA}
}

\renewcommand{\shortauthors}{Boaz}

\begin{abstract}
  npm (Node Package Manager) is a popular package manager owned by Microsoft that allows developers
  to share node packages and libraries. This article will review and analyze the design flaw of npm
  and how it was broken by a single package. This article will also review how Microsoft responded to
  the incident and what actions they took to prevent it from happening again.
\end{abstract}

\keywords{npm, design flaws, everything, left-pad}

\begin{CCSXML}
  <ccs2012>
  <concept>
  <concept_id>10003456.10003462.10003561.10003563</concept_id>
  <concept_desc>Social and professional topics~Network access restrictions</concept_desc>
  <concept_significance>300</concept_significance>
  </concept>
  <concept>
  <concept_id>10003456.10003462.10003561.10003566</concept_id>
  <concept_desc>Social and professional topics~Universal access</concept_desc>
  <concept_significance>100</concept_significance>
  </concept>
  <concept>
  <concept_id>10003456.10003462.10003574.10003000</concept_id>
  <concept_desc>Social and professional topics~Social engineering attacks</concept_desc>
  <concept_significance>500</concept_significance>
  </concept>
  <concept>
  <concept_id>10003456.10003462.10003574.10003578</concept_id>
  <concept_desc>Social and professional topics~Malware / spyware crime</concept_desc>
  <concept_significance>100</concept_significance>
  </concept>
  <concept>
  <concept_id>10002978.10002991</concept_id>
  <concept_desc>Security and privacy~Security services</concept_desc>
  <concept_significance>500</concept_significance>
  </concept>
  <concept>
  <concept_id>10002978.10003006</concept_id>
  <concept_desc>Security and privacy~Systems security</concept_desc>
  <concept_significance>500</concept_significance>
  </concept>
  </ccs2012>
\end{CCSXML}

\ccsdesc[500]{Security and privacy~Security services}
\ccsdesc[500]{Security and privacy~Systems security}
\ccsdesc[300]{Social and professional topics~Network access restrictions}
\ccsdesc[500]{Social and professional topics~Social engineering attacks}
\ccsdesc[100]{Social and professional topics~Universal access}
\ccsdesc[100]{Social and professional topics~Malware / spyware crime}

\maketitle

\fancyfoot{}
\thispagestyle{empty}

\section{Article Introduction}
Design flaws in software can sometimes lead to security breaches. More often than
not, it is these types of breaches that are the most popularized. However, sometimes
a design flaw can lead to a catastrophic failure that was not even intended
(or initiated) by a bad actor.
With over 1.3 million packages as of April 14th, 2021 \cite{npmblog-stats}, npm is
the world's largest software registry \cite{aboutnpm,w3schools}. As a
result, it has been heavily relied upon by developers for years.

Needless to say, the npm registry is a critical piece of infrastructure for the
entire JavaScript community (and by extension 99\% of websites \cite{w3techs}). Developers
expected that the npm registry would be reliable, secure, and complete. However, the
left-pad incident in 2016 \cite{npmblog-leftpad} and the everything package incident
in 2023 \cite{youtube-everything} have shown that the npm registry is not as reliable
as developers had hoped.

\section{Historical Background}
\subsection{Introduction}
One of the most infamous events in the history of npm was the left-pad incident.
Azer Koçulu is a software developer who was working on a personal project called
kik \cite{qz}. Coincidentally, Kik was also the name of a company in
Ontario, Canada \cite{qz, crunchbase}. The company reached out to Koçulu
requesting that he change the name of his project. Koçulu refused, and the company
proceeded to file a complaint with npm \cite{qz}. npm then decided to transfer the name
kik to the company, which caused Koçulu to remove all of his packages from the
npm registry, stating

"I want all my modules to be deleted including my account, along with this package.
I don't wanna be a part of NPM anymore. If you don't do it, let me know how do it quickly.
I think I have the right of deleting all my stuff from NPM." \cite{medium-mikeroberts}

\subsection{The left-pad package}
As a result of Koçulu's deciscion, all of his packages (including left-pad) were removed
from the npm registry. Despite being only 17 lines of code \cite{github-leftpad-code},
millions of packages relied on left-pad \cite{github-leftpad-dependency}. As a result,
many packages that directly or indirectly relied on left-pad broke, including React,
Babel, and Atom.

After an attempt to publish left-pad under a new version failed (due to some major
dependencies relying on the old version specificially, which was no longer available),
npm resulted to restoring the package from a backup \cite{npmblog-leftpad}. The total
downtime taken to restore the backup was 2.5 hours \cite{npmblog-leftpad}.

With the first ever unprecedented un-unpublishing of a package, npm was able to restore
the left-pad package and the packages that relied on it.

\subsection{Mitigation and Prevention}
Once the left-pad incident was resolved, npm published a blog post detailing the
incident and the steps they planned to take to prevent it from happening again \cite{npmblog-leftpad}.

`We will make it harder to un-publish a version of a package if doing so would break other packages.
We are still fleshing out the technical details of how this will work. Like any registry change,
we will of course take our time to consider and implement it with care.' \cite{npmblog-leftpad}

npm ended up enacting two policies relating to unpublishing packages. The first policy applied on
packages that were less than 72 hours old, allowing unpublishing as long as `no other packages
in the npm Public Registry depend on your package' \cite{npm-docs-unpublishing}.

The second policy applied to packages that were older than 72 hours, allowing unpublishing as long
as three conditions were met \cite{npm-docs-unpublishing}:
\begin{enumerate}
  \item no other packages in the npm Public Registry depend on it
  \item it had less than 300 downloads over the last week
  \item it has a single owner/maintainer
\end{enumerate}

With these new policies in place, npm hoped to prevent a similar incident from happening again.
However, these new policies lead towards a new design flaw.

\section{The everything package}
\subsection{Everything's Creation}
In 2023, a developer by the name PatrickJS decided to work on a package called `everything'.
He teamed up with Evan Boehs, another developer to work on creating an npm package that
would declare every single package in the npm registry as a dependency. This invovled
working around some obscure npm limitations, such as the maximum number of dependencies
being around 800 \cite{youtube-everything}, and the maxiumum package size being 10 MB \cite{youtube-everything}.

In order to work around these limitations, PatrickJS and Evan Boehs decided to chunk their
dependencies into smaller packages \lstinline|@everything-registry/chunk-0| through to \\
\lstinline|@everything-registry/chunk-4|. At the time there was approximately 2.5 million
packages in the npm registry \cite{youtube-everything}, so sub-chunking was also required.

Unsurprisingly, npm also has a rate limit on how quickly packages can be published. Thus,
PatrickJS and Evan Boehs used a GitHub Action workflow to publish the packages \cite{youtube-everything}.

\subsection{The Aftermath}
Upon publishing of the `everything' package in Dec 2023 \cite{npm-everything},
PatrickJS and Evan Boehs had (unintentionally) disabled unpublishing every package ever on
the npm registry \cite{youtube-everything}. This was an unforseen consequence of the new
unpublishing policies that npm had put in place after the left-pad incident. Namely,
\textit{every package was now a dependency of the `everything' package},
and thus, could not be unpublushed, regardless of age. \lstinline|everything| itself could
not be removed, as it had one dependency: \lstinline|everything-else| \cite{npm-everything-else}.

This was also caused by the fact that the `everything' package depended on a `*' version
of all of its packages, which meant that it matched any version of each of its packages.

\subsection{Historic Related Incidents}
npm was not a strange to packages and projects before `everything'.

\subsubsection{no-one-left-behind}
was an npm package that was created in 2018 that similarly depended
on 1,000 npm packages \cite{npm-no-one-left-behind-dependencies}. npm managed to remove
the package before it caused significant damage \cite{npm-no-one-left-behind-security}.
It was then re-uploaded under a different name, this time with 33,000 dependencies \cite{socket-everything}.

\subsubsection{hoarders}
was a package that was created near the early conception of npm (2012) that also depended
on 11,000 packages at its inception \cite{github-hoarders-dependencies}. Isaac Schluter (
the founder of npm) talked with the creator of the package and convinced him to unpublish it
\cite{github-hoarders}.  A few years later in Aug 19, 2021



\bibliographystyle{unsrtnat}
\bibliography{bib}

\appendix
\section{Reading Resources}
Azer Koçulu himself wrote a blog post about the incident, which can be found at \cite{medium-azer}.

\end{document}
\endinput